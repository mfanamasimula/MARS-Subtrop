
\documentclass[runningheads,a4paper]{article}

\usepackage[utf8]{inputenc}
 
\setcounter{tocdepth}{3}

\usepackage[english]{babel} 
\usepackage{graphicx}
\usepackage{grffile}
\usepackage{float}
\usepackage{multicol}
\usepackage{url}
 
\usepackage{titling}
\usepackage[hidelinks]{hyperref}
\setcounter{secnumdepth}{5}
%Margins
\usepackage[
margin=2cm,
includefoot
]{geometry}


\graphicspath{{img/}}

%Headers and Footers
\usepackage{fancyhdr}
\pagestyle{fancy}
\fancyhead{}
\fancyfoot{}
\fancyfoot[R]{\thepage}
\renewcommand{\headrulewidth}{0pt}
\renewcommand{\footrulewidth}{0pt}
 \setlength\parindent{24pt}
\begin{document}

	%Title Page
	\begin{titlepage}
		\begin{center}
			\includegraphics[width=10cm]{UP.jpg}  \\
			[1cm]
			\line(1,0){300} \\
			[0.3cm]
			\textsc{\Large
				Subtrop/OptiCrop\\
				Software Requirements Specification\\
			\hfill \break 
				%University of Pretoria
			}\\
			[0.1cm]
			\line(1,0){300} \\
			[0.7cm]
			\textsc{\Large
				Team MARS
			} \\
			
			
			
		\end{center}
		
		\begin{center}
			\begin{multicols}{2}
				\textsc{\large\\
				Banele Nxumalo\\ 
					12201911\\ 
				}
				
				\textsc{\large\\
				Mathapelo Matabane\\
					 15031625\\ 
				}
				
			
				
				\columnbreak
				
				\textsc{\large\\
					 Mfana Masimula\\
					 12077713\\ 
					}
				
				
				\textsc{\large\\
					Brian Ndung'u\\
					15322913\\
				}
				
			\end{multicols}
			
			
			\textsc{	\\ \href{https://github.com/mfanamasimula/MARS-Subtrop}{GitHub}
				\url{https://github.com/mfanamasimula/MARS-Subtrop.git}}
			
		\end{center}
	\end{titlepage}
%\maketitle

\begingroup

\tableofcontents
\addcontentsline{toc}{section}{Table Of Contents}
\endgroup
\newpage


\section{Introduction}

\subsection{Purpose}
The purpose of this Software Requirements Specification Document is to lay out the findings of a standardised requirements elicitation for the proposed application. In doing this, both the capabilities that the application is expected to deliver together, with the constraints on the solution space are described.\newline \newline The document will take into account requirements, as set out by both the developers and the prospective users, in order to lay a foundation for the up-coming phases of the Software Development Cycle. The various sections each provide details on specific types of requirements of the application.\newline \newline The intended audience include developers/peers, domain experts(COS 301 Lectures and the Subtrop Management),Subtrop(client),end-users(farmers, growers and their foreman) who will perform a technical, expert  and customer review of this document respectively.
 
\subsection{Scope}
The name of the proposed software application is OptiCrop. OptiCrop is intended to be an productivity application that is to enable growers and their foreman to measure yield, by combining a clocking system with yield data. It must be able to make use of GPS dat and based on weight and location assumptions, it should also be able to give approximate yield estimates not only for each orchard but for each approximate location where data was entered.Farmers must be able to enter and save the
different block / orchard names with detailed information such as cultivar, year planted etc on the farm. Farmers must be able to track the foremen in real time in the luxury of their office using the phone’s
GPS in order to determine where the workers are, without having to contact them. Show a path of where the phone had been for the past day / week etc could be displayed. A web interface will be necessary, especially to do administrative tasks such as adding worker’s names, block
names and details. Finally, the functional applicationshould be available on the Google Playstore
and/or the Appstore.\newline \newline The goal is to integrate existing methodoly which is currently being used on the farms and develop a fully functional mobile application. Another goal is for the application to make use of human resources for its development and usage i.e. expertise from the COS 301 lectures to satisfy the former and make use of crowd-sourcing principles (using smart devices) to serve the latter. \newline \newline The benefits include smoothness of the day-to-day operations on the farms which leads to greater productivity and efficiency among growers  and foreman  alike. It allows the farmers to be able to get correct data of the crop and be able to manage employees more easily.Also increase its brand value in the global market.

\subsection{Definitions, Acronyms, and Abbreviations}
\paragraph{\textbf{Acronyms}}
\begin{enumerate}
	\item Wi-Fi - Wireless Fidelity 
	\item SHA - Simple Hashing Algorithm
	\item UI - User Interface
	\item UX - User Experience
	\item GPS - Global Positioning System
	\item AP - Access points
	\item SRS - Software Requirements Specification
	\item IEEE - Institute of Electrical and Electronic Engineers 
	\item iOS - iPhone Operating System 
	\item IT - Information Technology
	\item 2D - Two Dimensional
	\item 3D - Three Dimensional
\end{enumerate}

\paragraph{\textbf{Abbreviations}}
\begin{enumerate}
	\item STD - Standard
	\item App - Application
	\item Info - Information
\end{enumerate}

\subsection{References}
Kung, D.C. (2013) Object-oriented software engineering: An agile unified methodology. New York: McGraw Hill Higher Education.

\subsection{Overview}
The rest of this document will be solely focussed on elaborating on the requirements of the OptiCrop application. It is structured according to the IEEE STD 830-1998 standard. An overall description of the product will be followed by a specific detailed requirements description. Each item in both the sections represent the ideas as generated during compilation and there is possibilities of change during the implementation of the application.

\section{Overall Description}

\subsection{Product Perspective}

\subsubsection{System Interfaces}
\begin{itemize}
	\item 
\end{itemize}
\subsubsection{User Interfaces}
\begin{itemize}
	\item 	There will be a directions icon on the home screen to seek directions to specific crop
	\item 	Icon to specify where you are on the map
	\item Compass to show which direction you’re currently facing
	\item The map will be take up majority of the screen(background)
	\item Menu/options list which will show all options for the map such as displaying heat maps and logging in of different types of users
	
\end{itemize}
\subsubsection{Hardware Interfaces}
\begin{itemize}
	\item	NavUP is intended to be a mbile application for both the Andriod and iOS platform.
	\item Both iOS and Andriod provide abstractions for all network communications and hardware.
	
	
\end{itemize}
\subsubsection{Software Interfaces}
\begin{itemize}
	\item OptiCrop applications will use the Java JDK and the Android SDK tools. For iOS Objective-C or Xamarin will be used.
	
\end{itemize}
\subsubsection{Communication Interfaces}
\begin{itemize}
	\item 	The OptiCrop application could use a HTTP server which uses a push protocol to push notifications onto the device
	\item 
	
\end{itemize}
\subsubsection{Memory}
\begin{itemize}
	\item For Android  it should work be able to work on Android 4.23.0 to the latest version which is 7.0. And for iOS is should be able to work from 7.0 and later.
	
\end{itemize}
\subsubsection{Operations}
\begin{itemize}
	\item Measure yield
	\item Show different blocks/orchards
	\item Navigate
	\item	Determine current location
	\item	Show heatmap
	\item	Track workers location
	\item	Provide directions
	
\end{itemize}
\subsubsection{Site Adaptation Requirements}

\subsection{Product Functions}
\paragraph{...}
\subsection{User Characteristics}
\paragraph{...}
\subsection{Constraints}
\paragraph{...}

% BEGIN Avinash
\subsection{Assumptions and Dependencies}
%\let\labelitemi\labelitemii
The factors that affect the requirements are:

% END Avinash


\section{Specific Requirements}

\subsection{External Interface Requirements}
\paragraph{...}
\subsection{Functional Requirements}

\subsubsection{User Login}
\subsection{Other Requirements}
\paragraph{...}

\section{Appendixes}
\paragraph{...}

\section{Index}
\paragraph{...}

\end{document}
